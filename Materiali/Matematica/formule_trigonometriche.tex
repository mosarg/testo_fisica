\documentclass[a4paper,10pt,fleqn,oneside]{article}

\usepackage[latin1]{inputenc}
\usepackage{amsfonts}
\usepackage{amsmath}
\usepackage{amscd}
\usepackage[italian]{babel}
\usepackage[T1]{fontenc}



\date{19/9/2006}

\begin{document}
\noindent
{\Large Formule Trigonometriche}\footnote{Written with \LaTeXe}\\
\vspace{1cm}

Formule di Somma
\begin{equation}
\cos(\alpha +\beta)=\cos\alpha\cos\beta -\sin\alpha\sin\beta 
\end{equation}
\begin{equation}
 \cos(\alpha -\beta)=\cos\alpha\cos\beta +\sin\alpha\sin\beta
\end{equation}
\begin{equation}
 \sin(\alpha +\beta)=\sin\alpha\cos\beta+\cos\alpha\sin\beta
\end{equation}
\begin{equation}
 \sin(\alpha -\beta)=\sin\alpha\cos\beta-\cos\alpha\sin\beta
\end{equation}
\begin{equation}
 \tan(\alpha +\beta)=\frac{\tan\alpha +\tan\beta}{1-\tan\alpha\tan\beta}
\end{equation}
\begin{equation}
 \tan(\alpha -\beta)=\frac{\tan\alpha -\tan\beta}{1+\tan\alpha\tan\beta}
\end{equation}
\vspace{1cm}

Formule di Dupplicazione
\begin{equation}
 \sin 2\alpha=2\sin\alpha\cos\alpha
\end{equation}
\begin{equation}
 \cos 2\alpha=\cos^2\alpha -\sin^2\alpha
\end{equation}
\begin{equation}
 \tan 2\alpha=\frac{2\tan\alpha}{1-\tan^2\alpha}
\end{equation}
\vspace{1cm}

Formule di Bisezione
\begin{equation}
 \sin^2\frac{\alpha}{2}=\frac{1-\cos\alpha}{2}
\end{equation}
\begin{equation}
 \cos^2\frac{\alpha}{2}=\frac{1+\cos\alpha}{2}
\end{equation}
\begin{equation}
 \tan^2\frac{\alpha}{2}=\frac{1-\cos\alpha}{1+\cos\alpha}
\end{equation}
\begin{equation}
 \tan\frac{\alpha}{2}=\frac{\sin\alpha}{1+\cos\alpha}\ \wedge\ \alpha\neq(2\kappa +1)\pi
\end{equation}
\begin{equation}
 \tan\frac{\alpha}{2}=\frac{1-\cos\alpha}{\sin\alpha}\ \wedge\ \alpha\neq\kappa\pi
\end{equation}























\end{document}
