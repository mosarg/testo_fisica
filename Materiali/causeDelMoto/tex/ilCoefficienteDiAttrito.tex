\subsubsection*{Il coefficiente di attrito}

La seconda forza che agisce sul cilindro che rotola è l'attrito. Questa forma di attrito è detta attrito volvente, perché ha la funzione di far ruotare il cilindro su se stesso e, al medesimo tempo, di rallentarne la corsa.\newline

L'attrito tra due corpi è tanto maggiore quanto più essi sono compressi l'uno contro l'altro, e viene pertanto confrontato con la forza di reazione vincolare esercitata dal piano inclinato sul cilindro mobile. Si definisce pertanto come coefficiente d'attrito il rapporto:

\begin{center}
\begin{equation}
\mu = \frac {a_a}{a_{\perp}}
\end{equation}
\end{center}
