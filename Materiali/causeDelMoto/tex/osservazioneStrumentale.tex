\subsubsection*{La descrizione strumentale}

Non tutte le osservazioni preliminari effettuate immediatamente risultavano concordi. Alcune, addirittura, erano in piena contraddizione. Abbiamo perciò sentito la necessità di individuare degli strumenti di misura capaci di fornire dei dati oggettivi affidabili.\newline

Abbiamo perciò deciso di riprendere il movimento facendo uso delle webcam dei nostri computer portatili o dei nostri cellulari, per realizzare successivamente un'analisi in laboratorio di informatica.\newline

Inizialmente, abbiamo disposto un metro da sarta centimetrato lungo il banco, ma ci siamo accorti che la lettura sulle telecamere risultava piuttosto incerta, a causa della bassa risoluzione del metro, o forse a causa della nostra limitata perizia di cameramen. Di conseguenza, abbiamo preferito tracciare sul banco alcune linee trasversali con dei pennarelli non indelebili.

I filmati sono stati scaricati in laboratorio di informatica e scomposti in fotogrammi con l'uso del programma libero ffmpeg.\newline
Per ogni filmato è stata creata una apposita cartella di fotogrammi, numerati in ordine cronologico. Individuando il numero dei fotogrammi in cui la pila toccava le linee trasversali, era possibile ottenere dei grafici spazio tempo che sono stati successivamente analizzati con geogebra.
