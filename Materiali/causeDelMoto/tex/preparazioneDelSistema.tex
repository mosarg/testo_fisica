\subsubsection*{Il rotolamento di un cilindro}

Come mostra l'immagine in apertura di capitolo, l'attività su cui è basata questa unità didattica riguarda il rotolamento di un cilindro (nello specifico una pila commerciale) lungo un banco di un laboratorio di fisica e descrive un esperimento realmente realizzato in una scuola superiore.\newline

Il banco era liscio e aveva la lunghezza di un paio di metri. La pila è stata fatta rotolare più volte, con spinte diverse. {\slshape Sorprendentemente}, in ogni lancio, il movimento della pila si è prolungato al di là di ogni attesa, terminando sulla parete in fondo al tavolo, anche nei casi in cui la velocità iniziale era molto piccola, dando quasi la sensazione che il moto della potesse prolungarsi all'infinito.\newline
Allora abbiamo provato a spingere la pila nella direzione opposta, dalla parete verso il lato libero del tavolo. Il moto è apparso estremamente diverso: adesso la pila rallentava visibilmente e di solito si fermava sul banco. Era proprio necesario imprimere una spinta molto robusta per riuscire a percorrere l'intera lunghezza del tavolo.\newline

Dove aver eseguito un certo numero di esperienze, ci siamo disposti a documentarle utilizzando un foglio suddiviso un foglio in tre colonne corrispondenti alle voci seguenti:\newline

\begin{center}
\begin{tabular}{|c|c|c|}
\hline
{\bf Cosa faccio} &{\bf Cosa osservo }&{\bf Come spiego }\\
\hline
\end{tabular}
\end{center}

