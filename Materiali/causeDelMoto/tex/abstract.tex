La conoscenza dei concetti di spazio, velocità e accelerazione è una premessa fondamentale per lo studio della fisica, perché i fenomeni naturali sono eventi dinamici di cui è necessario saper analizzare i mutamenti.\newline
Però la fisica si pone un obiettivo più alto, che trascende la cinematica. Non è sufficiente, infatti, riconoscere le caratteristiche di un fenomeno, ma è necessario sviluppare la capacità di ricavare {\bf le cause} di ciò che si osserva.\newline

In questo capitolo, partendo dall'osservazione di un fenomeno cinematico estremamente semplice, cercheremo di distinguere il ruolo della gravità e dell'attrito come cause del moto e riflettere su alcuni caratteri dello scivolamento lungo un piano inclinato.\newline

Siccome la trattazione si appoggia a un'attività di laboratorio realmente svolta in una scuola superiore, quanto segue sarà esposto, in alcune parti, secondo la forma documentale della relazione di laboratorio.
