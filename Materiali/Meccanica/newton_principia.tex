\documentclass[a4paper,10pt,oneside]{article}
\usepackage[polutonikogreek,italian]{babel}
\usepackage[utf8x]{inputenc}
\usepackage{amsmath}
\usepackage{amsthm}
\usepackage{amssymb}
\usepackage{amscd}
\usepackage{graphicx}
\usepackage{float}
\usepackage{array}
\usepackage{rotating}
\usepackage[small]{caption}
\usepackage{lscape}
\usepackage{fancybox}
\usepackage{booktabs}
\usepackage[noanswer]{exercise}
\parindent0ex
\renewcommand{\fboxsep}{0.4cm}
\usepackage{hyperref}
\renewcommand{\textfraction}{0.05}
\renewcommand{\topfraction}{0.95}
\renewcommand{\bottomfraction}{0.95}
\renewcommand{\floatpagefraction}{0.35}
\renewcommand{\ExerciseName}{Esercizio}
\renewcommand{\ExerciseListName}{Es}
\setcounter{totalnumber}{5}
\restylefloat{figure}
\begin{document}
\thispagestyle{empty}
 \section*{Il secchio rotante: il punto di vista di Newton}

\vspace{1cm}

Gli effetti per i quali i moti assoluti e relativi si distinguono gli uni dagli altri, sono le forze di allontanamento dall'asse del moto circolare. Infatti nel moto circolare puramente relativo queste forze sono nulle, mentre nel moto vero e assoluto sono maggiori o minori a seconda della quantità di moto. Si sospenda un recipiente ad un filo abbastanza lungo, e si agisca con moto circolare fino a che il filo, a causa della tensione, si indurisca completamente. Si riempia il recipiente di acqua e lo si faccia riposare insieme con l'acqua; lo si muova, poi, con forza subitanea in senso contrario lungo un cerchio; allora allentandosi il filo, continuerà a lungo in questo moto. All'inizio la superficie dell'acqua sarà piana, come prima del moto del vaso; e poiché il vaso comunicata gradualmente la forza all'acqua, fa in modo che anche questa inizi più sensibilmente a ruotare, l'acqua comincerà a ritirarsi a poco a poco dal centro e salirà verso i lati del vaso, formando una figura concava (come io stesso ho sperimentato)[\ldots] All'inizio quando il moto relativo dell'acqua nel vaso era massimo, quello stesso moto in nessun modo eccitava lo sforzo di allontanamento dall'asse; l'acqua non tendeva alla circonferenza con l'ascendere verso i lati del vaso, ma rimaneva piana, e perciò non era ancora iniziato il suo vero moto circolare. Dopo, diminuito il movimento relativo dell'acqua, la sua ascesa verso le pareti del vaso indicava lo sforzo di allontanamento  dall'asse del moto e questo sforzo indicava che il suo vero moto circolare cresceva continuamente fino al punto massimo in cui l'acqua giaceva in quiete relativa nel vaso[\ldots] È difficilissimo in verità conoscere i veri moti dei singoli corpi e distinguerli di fatto dagli apparenti: e ciò perché le parti dello spazio immobile, in cui i corpi veramente si muovono non cadono sotto i sensi. La cosa tuttavia non è affatto disperata. Gli argomenti, infatti, possono essere desunti in parte dai moti apparenti, che sono le differenze dei moti veri, in parte dalle forze che sono cause ed effetti dei moti veri. Cosicché, se due globi, legati ad un filo ad una determinata distanza l'uno dall'altro, vengono fatti ruotare attorno al comune centro di gravità, si conoscerà dalla tensione del filo, lo sforzo di allontanamento dei globi dall'asse del loro movimento, e di conseguenza si potrà calcolare la quantità del movimento circolare. Inoltre, se, al fine di aumentare o diminuire il moto circolare, si applicassero simultaneamente forze uguali qualsiasi, ora sull'una faccia ora sull'altra dei globi, dalla aumentata o diminuita tensione del filo si potrebbe conoscere l'aumento o il decremento del moto,e allora, infine, si potrebbe stabilire su quali facce dei globi le forze dovrebbero essere applicate per aumentare al massimo il movimento; ossia le facce più lontane, vale a dire quelle che nel moto circolare seguono. Una volta conosciute le facce che seguono, e le facce opposte che precedono, verrà conosciuta la determinazione del moto. In questo modo potrebbe venir trovata la quantità e la determinazione di questo moto circolare  in qualunque vuoto immenso, ove non esiste alcunché di esterno e sensibile con cui i globi potrebbero essere confrontati\ldots

\vspace{1cm}

{\small Tratto da Isaac Newton \emph{Principi Matematici} a cura di Alberto Pala Torino UTET 1965}

\end{document}
