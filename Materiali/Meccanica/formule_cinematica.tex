\documentclass[a4paper,10pt,oneside]{article}
\usepackage[polutonikogreek,italian]{babel}
\usepackage[utf8x]{inputenc}
\usepackage{amsmath}
\usepackage{amsthm}
\usepackage{amssymb}
\usepackage{amscd}
\usepackage{graphicx}
\usepackage{float}
\usepackage{array}
\usepackage{verbatim}
\usepackage{rotating}
\usepackage[small]{caption}
\usepackage{lscape}
\usepackage{fancybox}
\usepackage{booktabs}
\parindent0ex
\renewcommand{\fboxsep}{0.5cm}
\usepackage{hyperref}
\renewcommand{\textfraction}{0.05}
\renewcommand{\topfraction}{0.95}
\renewcommand{\bottomfraction}{0.95}
\renewcommand{\floatpagefraction}{0.35}
\setcounter{totalnumber}{5}
\restylefloat{figure}
\newlength{\drop}
\begin{document}
\section*{Cinematica unidimensionale}


\section*{Definizioni}

Definiamo spostamento la variazione della posizione (in un ben definito sistema di riferimento) di un punto materiale:
\begin{equation}\label{spostamento}
\Delta x=x(t)-x(t_0)\qquad t>t_0
\end{equation}
nella [\ref{spostamento}] $x(t)$ rappresenta la coordinata posizione dell'oggetto nel sistema di riferimento unidimensionale in cui si trova il punto materiale.
Definiamo velocità vettoriale (\emph{velocity}) media il rapporto tra spostamento ed intervallo temporale in cui tale spostamento è avvenuto:
\begin{equation}\label{vmedia}
 \bar{v}=\frac{\Delta x}{\Delta t}
\end{equation}
Definiamo velocità scalare (\emph{speed}) media il rapporto tra lo spazio percorso e il tempo in cui tale spazio è stato percorso:
\begin{equation}
 v_{speed}=\frac{<spazio>}{\Delta t}
\end{equation}

Definiamo velocità vettoriale istantanea il valore verso cui ``tende'' la velocità vettoriale media al ``tendere'' a zero dell'intervallo temporale $\Delta t$:
\begin{equation}
 v(t)=\lim_{\Delta t \to 0} \frac{\Delta x}{\Delta t}
\end{equation}
Definiamo accelerazione media il rapporto tra la variazione di velocità vettoriale e l'intervallo di tempo in cui questa variazione avviene:
\begin{equation}\label{amedia}
 \bar{a}=\frac{\Delta v}{\Delta x}
\end{equation}
Definiamo accelerazione istantanea il valore verso cui ``tende'' l'accelerazione media al ``tendere'' a zero dell'intervallo temporale $\Delta t$:
\begin{equation}
 a(t)=\lim_{\Delta t \to 0}\frac{\Delta v}{\Delta t}
\end{equation}
\section*{Leggi orarie}
In classe abbiamo studiato tre degli infiniti modi in cui un punto  materiale può muoversi lungo una dimensione spaziale:
\begin{enumerate}
 \item Il corpo non varia la sua posizione nel tempo
 \item Il corpo si muove con velocità costante
 \item Il corpo si muove con accelerazione costante
\end{enumerate}
\subsection*{Il corpo non varia la sua posizione nel tempo}
Se un punto materiale di cui stiamo descrivendo un moto unidimensionale non varia la sua posizione nel tempo la sua coordinata $x$ deve essere costante, possiamo quindi scrivere la legge oraria come:
\begin{equation}
 x(t)=k
\end{equation}
dove $k$ non varia nel tempo
\subsection*{Il corpo si muove con velocità costante}
Se il corpo non varia la sua velocità nel tempo, dalla definizione di velocità media deduciamo che:
\begin{equation}
 \bar{v}=v
\end{equation}
ovvero la velocità media è uguale alla velocità istantanea. Possiamo quindi riscrivere la [\ref{vmedia}] usando la [\ref{spostamento}] come:
\begin{equation}
 x(t)=x(t_0)+vt
\end{equation}
da cui risulta evidente come il corpo percorra spazi uguali in tempi uguali.
\subsection*{Il corpo si muove con accelerazione costante}
In questo caso utilizziamo la relazione [\ref{amedia}] e siccome per definizione l'accelerazione è costante quella media risulterà uguale a quella istantanea:
\begin{equation}\label{vacc}
 v(t)=v_0+a t
\end{equation}
La relazione [\ref{vacc}] illustra la relazione lineare che intercorre tra velocità e tempo. Ricordiamo che in un moto uniformemente accelerato, a causa della natura lineare succitata della relazione tra velocità e tempo, la \emph{velocità media} è uguale alla \emph{media delle velocità}\footnote{Chiaramente questo \textbf{non} è vero in generale}:
\begin{equation}\label{vmedia2}
 \bar{v}=\frac{1}{2}\left(v(t)+v(t_0)\right)
\end{equation}
combinando quindi la [\ref{vmedia2}], [\ref{vmedia}] e la [\ref{vacc}]   otteniamo la legge oraria del moto uniformemente accelerato:
\begin{equation}\label{acc}
 x(t)=x_0+v_0  t +\frac 1 2 a t^2
\end{equation}
utilizzando la [\ref{vacc}] e la [\ref{acc}] è possibile ottenere della altre utili relazioni:
\begin{equation}
 v^2=v_0^2+2a(x-x_0)
\end{equation}
in cui non compare il tempo
\begin{equation}
 x-x_0=\frac 1 2(v+v_0)t
\end{equation}
in cui non compare l'accelerazione
\begin{equation}
 x-x_0=vt-\frac 1 2 a t^2
\end{equation}
in cui non compare la velocità iniziale

\end{document}
