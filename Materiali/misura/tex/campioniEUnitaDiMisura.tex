\subsubsection*{Campioni e unità di misura}
$

$
Una misura è un'operazione di confronto.
$

$
Per esempio, la misura di superficie all'inizio di questo capitolo era
$
il confronto tra la dimensione di un quadretto di quaderno. Nel nostro esercizio, il quadretto era un campione che
$
svolgeva la funzione di unità di misura.
$
\newline
$

$
Concettualmente, la scelta dell'unità di misura è attribuita alla libera scelta dell'operatore che esegue la misura ma,
$
agli effetti pratici, è necessario fare riferimento a delle convenzioni riconosciute. La scelta dell'unità di misura,
$
in questo caso, è deferita alle disposizioni di un'autorità riconosciuta.
$
\newline
$

$
Quando una grandezza viene confrontata con il campione, si esegue una misura diretta.
$
Sempre in riferimento all'attività del primo paragrafo,
$
proviamo a descrivere una ad una le singole operazioni necessarie
$
per realizzare una misura diretta, come se dovessimo generare un algoritmo per
$
programmare un computer. Forse potremmo operare così:
$
\begin{enumerate}
$
\item Produciamo una superficie di prova\( ~\mathcal{P}\);
$
\item Confrontiamo la superficie $~\mathcal{S}$ ad con un singolo quadretto;
$
\item Se la superficie è maggiore del quadretto aggiungiamo un quadretto a
$
$~\mathcal{P}~$ e riportiamoci al punto 2);
$
\item altrimenti poniamo \(~\mathcal{S}~ = ~\mathcal{P}~ \) ed usciamo.
$
\end{enumerate}
$

$
In generale, quando si esprime il risultato di una misura, si assume sempre di
$
avere eseguito una vera e propria misura diretta.
$
\newline
$

$
Ma, nella pratica, non sempre è cos\'i.
$
\newline
$
Per esempio, la distanza tra la terra e il Sole è pari a $~150~10^{12}~m~$
$
(centocinquantamila miliardi di metri !). Il metro campione è conservato
$
all'Ufficio internazionale pesi e misure di Sevres, vicino a Parigi, ed è lungo
$
circa un paio di gomiti. Non vorrete pensare di mandarci il vostro compagno di banco,
$
a misurare in modo diretto la distanza terra-Sole, con il metro campione di
$
Parigi ...
$
\newline
$

$
Il campione di misura è detto anche {\bf unità di misura}. Per
$
convenzione, la comunità scientifica ha selezionato alcune unità di
$
misura convenzionali\footnote{secondo un meccanismo determinato dal cosiddetto
$
Sistema Internazionale di unità di misura, che è descritto al paragrafo successivo}
$
per tutte le grandezze d'uso più
$
comune. Tuttavia è spesso opportuno, per facilitare la comprensione dei fenomeni
$
studiati, utilizzare dei multipli o dei sottomultipli dell'unità di misura di
$
riferimento.
$
\newline
$
Per esempio, per le lunghezze, l'unità di misura convenzionale è il metro, ma
$
spesso le misure di lunghezza sono espresse in chilometri, in ettometri, in
$
decametri, in centimetri, in millimetri, in micron e così via. Un chilometro
$
equivale a mille metri, e viene rappresentato con l'uso del prefisso {\bf k}:
$

$
\begin{center}
$
\begin{math}
$
1~km~=~10^3~m
$
\end{math}
$
\end{center}
$
Il millimetro corrisponde a un millesimo di metro (sottomultiplo), e viene
$
rappresentato con l'uso del prefisso {\bf m}:
$

$
\begin{center}
$
\begin{math}
$
1~km~=~10^{-3}~m
$
\end{math}
$
\end{center}
$
Ancora, il ${\bf Gm}$ (gigametro) corrisponde a un miliardo di metri, il ${\bf Mm}$
$
(megametro) a un milione di metri, il ${\mathbf \mu m}$ (micrometro) a un milionesimo di
$
metro (cioè un millesimo di millimetro) e il ${\bf nm}$ (nanometro) a un miliardesimo
$
di metro. Per una lista più ampia, consultate wikipedia
$
\footnote{
$
\url{http://it.wikipedia.org/wiki/Sistema_internazionale_di_unit\%C3\%A0_di_misura\#Prefissi}}
$
.
$
