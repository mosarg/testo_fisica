\subsubsection*{Misura non ripetibile di lunghezza}
$
Misurare una lunghezza è un'operazione che certamente abbiamo ripetuto molte volte ma, probabilmente, sempre
$
in un solo {\itshape modo}. In realtà, il modo migliore di misurare una lunghezza è sconosciuto quasi a tutti.
$

$
Eppure, è un metodo tanto semplice quanto sorprendente.
$
\newline
$

$
Per cominciare, procuratevi un mattarello o un piccolo rullo da cucina, che utilizzerete ad un tempo come strumento e
$
come unità di misura. Scegliete anche una distanza da misurare cercando una superficie piana accessibile. Per esempio
$
la lunghezza tra due tacche sul banco o sulla lavagna, l'altezza di un libro o quant'altro. Effettuate un primo
$
rilievo di test e discutetene.
$

$
In particolare, identificate:
$
\begin{itemize}
$
\item il valore della vostra misura, prendendo come unità la circonferenza del rullo;
$
\item l'errore assoluto;
$
\item l'errore relativo;
$
\end{itemize}
$
In più, descrivete il {\slshape dettaglio} delle operazioni compiute (senza eccedere nel prolisso) e definite la
$
riproducibilità della vostra misura.
$

$
Verificate in cosa il vostro modo di procedere si è distinto dai passi suggeriti qui sotto:
$
\begin{enumerate}
$
\item contrassegnare il rullo con una tacca stabile;
$
\item appoggiare con la massima attenzione la tacca su una delle due estremità della lunghezza da misurare;
$
\item rotolare lentamente il rullo, {\slshape senza strisciare};
$
\item contare il numero dei contatti con la superficie, durante il rotolamento.
$
\item Detto con {\bf n} tale numero, indicare come risultato della misura il valore:
$
\newline
$
\begin{center}
$
 \begin{math}n < L < n+1\end{math}
$
\end{center}
$
\end{enumerate}
$

$
Nulla di più ovvio. Eppure, come vedremo, nulla di più inefficiente.
$
\newline
$

$
Di sicuro avete risparmiato tempo e fatica, ma non avete imparato niente e, soprattutto, siete {\slshape obbligati} ad
$
accettare come precisione della vostra misura la sensibilità (manifestamente scadente) del vostro rullo.
$

$
Come rimediare? Semplicemente, ripensando con occhio critico il passo {\bf b)} della vostra procedura, che è il vero
$
punto debole ...
$
\newline
$

$
Prvate ad esempio, a ridefinirlo come segue:
$
\begin{itemize}
$
\item Appoggiate il rullo ad un'estremità del percorso da misurare in un modo del tutto casuale, usando tutte le cure
$
possibili per non influire sulla posizione di partenza della tacca.
$
\end{itemize}
$
Si può obiettare che, in questo modo, non è più possibile attendersi che la misura ripeta ogni volta lo stesso
$
risultato. Ma è esattamente questa la proprietà che distingue le misure {bf non ripetibili}.
$

$
Tuttavia, affermare che il risultato atteso di ogni misura successiva può essere diverso dal precedente, non significa
$
ottenere un grado completo di casualità\footnote{come se, ad esempio, si estraesse un numero a caso dal sacchetto della
$
tombola}. Supponete che, eseguendo in modo {\slshape diligente} le prime due misure, accada di ottenere una volta otto unità e una volta sette.
$
(cosa del tutto realizzabile, con questo modo di fare). Cosa si potrebbe attendere per la terza?
$
Che cosa pensereste se, a sorpesa, il terzo risultato fosse sei oppure nove?
$
\newline
$

$
Se siete {\slshape diligenti}, tutte le rilevazioni devono essere {\bf distribuite} su due valori interi consecutivi.
$
Eventuali violazioni di questa regola vanno attribuite a errori sistematici manifesti e scartate.
$
\newline
$

$
Ora, l'esecuzione di una misura non ripetibile si articola in due fasi:
$
\begin{enumerate}
$
\item Acquisizione di un numero congruo di dati, nel rispetto diligente della procedura prestabilita;
$
\item elaborazione dei dati.
$
\end{enumerate}
$
Perciò, appena vi sentirete ben sicuri di saper ripetere il processo, dividetevi a piccoli gruppi ed eseguite la prima
$
fase. Raccogliete almeno una cinquantina di dati per gruppo.
$
\newline
$

$
L'elaborazione dei dati è sempre un'attività di matematica. Gli algoritmi da usare cambiano di
$
volta in volta, a seconda del {\itshape fenomeno} osservato e del {\itshape metodo} di misura. In questo caso, la
$
matematica necessaria è quella della cosiddetta {\slshape ""distribuzione binomiale a due valori""}. Siccome,
$
probabilmente, non avete gli strumenti per capirla bene, applicate semplicemente le regole qui indicate:
$
\begin{enumerate}
$
\item come risultato della misura, calcolate la media aritmetica dei valori;
$
\item come come errore relativo della misura, calcolate il reciproco della radice quadrata del numero di misure
$
effettuate.
$
\end{enumerate}
$

$
Per esercizio, stabilite la distanza tra il portone di casa vostra e l'ingresso della scuola, con una precisione pari a
$
un decimo di circonferenza di una ruota di bicicletta.
$
