\subsubsection*{Le cifre significative e la notazione scientifica}
$
Un cubetto di rame di un centimetro cubo contiene, allineati in modo regolare su ciascuno spigolo, cento milioni di
$
atomi. In tutto, sono 1000000000000000000000000 atomi. Ciascuno di essi occupa un volume di 
$
0,000000000000000000000000000001 \begin{math}m^3\end{math}.
$

$
La fisica si interessa di fenomeni molto piccoli e di fenomeni molto grandi, e ha la necessità di scriverli in un modo
$
espressivamente efficace. Come nei due esempi qui sopra, la notazione decimale è particolarmente scomoda, appena si
$
evade dall'uso quotidiano dei prezzi al supermercato o dei problemi didattici dei nostri libri di testo. Già i
$
matematici, quando incontrano un numero importante, sono abituati a sostituirlo con un simbolo, perché la notazione
$
decimale è troppo pesante e di scarso interesse.
$

$
In fisica, un numero deve contemporaneamente, essere facile da leggere, da capire e da usare per scopi {\slshape
$
operativi}. Per questo, è necessario distinguere, nella scrittura, le parti importanti da quelle trascurabili,
$
attraverso un insieme di convenzioni e regole specifiche.
$
\newline
$

$
Un primo problema, è quello di trovare una convenzione capace di incorporare l'errore nella scrittura del risultato.
$

$
In precedenza, infatti, abbiamo descritto il metodo della forchetta, che è sicuramente il modo più efficace e corretto
$
di separare la stima del valor vero di una misura dall'errore associato. In certe situazioni, quando non serve essere
$
troppo rigorosi, si preferisce usare la convenzione delle cifre significative. Nell'esposizione di un
$
risultato\footnote{in notazione decimale} si scrivono solo quelle parti del numero che si ritiene di avere misurato con
$
sicurezza.
$
\newline
$

$
Supponiamo di effettuare una misura di lunghezza con un righello da cartoleria.
$
In questo caso la precisione della misura è pari al passo del righello, cioè un
$
millimetro. Allora indicheremo il risultato in questo modo:
$
\begin{center}
$
	\begin{math}
$
      		L_{righello}\;=~12,5~cm
$
	\end{math}
$
		oppure
$
	\begin{math}
$
		L_{righello}~125~mm
$
	\end{math}
$
\end{center}
$
Se, successivamente, dovessimo rivalutare la stessa misura con l'uso di un
$
calibro, e se ottenessimo esattamente lo stesso risultato, lo dovremmo scrivere
$
in un modo diverso, evidenziando la maggior precisione:
$

$
\begin{center}
$
	\begin{math}
$
		L_{calibro}~=~12,50~cm
$
	\end{math}
$
		oppure
$
	\begin{math}
$
		L_{calibro}~= ~ 125,0~ mm
$
	\end{math}
$
\end{center}
$

$
Un riferimento di buona qualità, per approfondire il concetto di cifre significative e per fare
$
qualche esercizio, si può trovare all'interno del sito web dell'Università di Messina\footnote{
$
\url{http://ww2.unime.it/cclchim/generale/significative.htm}}
$
\newline
$

$
Un secondo problema, legato alla scrittura di un numero, è quello di trasmettere a prima vista la percezione della
$
dimensione, affinché sia possibile confrontarlo rapidamente con altri numeri più grandi o più piccoli. Il volume di un
$
atomo, ad esempio, non deve {\bf mai} essere scritto come abbiamo fatto sopra, altrimenti risulta impossibile sia da
$
leggere che da usare.
$
Una tecnica molto utile è la {\bf notazione scientifica}. Ogni numero viene separato in due parti. La più importante è
$
senz'altro l'ordine di grandezza. Lo spazio occupato da un atomo in un reticolo cristallino, ad esempio, diventa
$
semplicemente \begin{math}10^{-30}~m^3\end{math} e la densità degli atomi sarà di \begin{math}10^{24}
$
atomi/cm^3\end{math}.
$

$
In questi esempi, gli zeri non sono cifre significative. Invece di scriverli tutti, è molto meglio contarli!
$
\newline
$

$
Consideriamo ora un esempio di come l'uso della notazione scientifica permette di superare alcune ambiguità che la
$
notazione decimale lascerebbe irrisolte, nell'indicazione delle cifre significative.
$
\newline
$

$
Immaginiamo di misurare
$
a passi la lunghezza di un campo di calcio. Supponiamo di avere misurato 100
$
passi e che un singolo passo possa essere stimato uguale a 1,1 metri, con una
$
precisione intorno al 10\%. Esprimendo il risultato con una forchetta potremmo
$
scrivere:
$

$
\begin{center}
$
	\begin{math}
$
		100~m~ <~ L~ <~ 120~ m
$
	\end{math}
$
\end{center}
$
Per come è stato posto il problema, noi siamo in grado di riconoscere, nella lunghezza del campo, i decametri, ma non i
$
metri. Se esprimiamo il risultato in metri, con la notazione decimale, le prime due cifre sono significative e la terza
$
no.
$

$
Se usiamo la notazione decimale, scrivendo \(L~= ~ 110~ m\), siamo costretti ad utilizzare una cifra
$
significativa in più del necessario e quindi dichiariamo un risultato {\slshape gravemente} (sic!)
$
errato.
$
\newline
$

$
 Per evitare questo problema, esistono essenzialmente due modi:
$

$
\begin{itemize}
$
\item Rinunciare all'uso del metro come unità di misura, sostituendo con un
$
multiplo adeguato. Per esempio:
$

$
\begin{center}
$
	\begin{math}
$
		L~ = ~ 0,11~ km
$
	\end{math}
$
\end{center}
$
\item Usare la notazione scientifica, scrivendo:
$
\begin{center}
$
	\begin{math}
$
		L~ = ~11~ *~ 10^1~ m
$
	\end{math}
$
\end{center}
$
\end{itemize}
$

$
Per approfondire il concetto di
$
notazione scientifica potete trovare moltissimi riferimenti. Proviamo qui a indicare la definizione di {\itshape
$
\href{http://www.wikipedia.org/wiki/Notazione_scientifica}{wikipedia}} e
$
discutetela con l'insegnante\footnote{n.d.a: per la verità, a tutto il 3 luglio 2011, quella definizione
$
\`e piuttosto carente. Ne proponiamo la lettura perché, a nostro parere dovrebbe essere rivista da qualche studente di
$
buona
$
volontà.}.
$
