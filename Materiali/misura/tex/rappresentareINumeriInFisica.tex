\subsection*{Rappresentare i numeri in fisica}
$
In fisica, i numeri hanno un significato molto diverso che in matematica.
$
\newline
$

$
Un autista sufficientemente attento può mantenere con facilità il veicolo ad una
$
velocità costante prefissata, ad esempio di 50 km/h, per un lungo intervallo di
$
tempo. Tuttavia, non è assolutamente detto che un sensore di velocità
$
indipendente, collocato sui bordi della strada, debba rilevare necessariamente
$
lo stesso identico valore di 50 km/h, al passaggio dell'autoveicolo.
$

$
Il motivo risiede nel fatto che il tachimetro dell'automobile, così come i
$
sensori di velocità commerciali d'uso comune, hanno una precisione che si
$
aggira intorno al 5\%, ma riportano il dato con una sensibilità superiore alle
$
proprie caratteristiche.
$
\newline
$

$
Quindi, pur leggendo il 50 sul proprio tachimetro, l'autista potrebbe
$
indifferentemente affermare di viaggiare a 48 km/h, piuttosto che a 52 km/h,
$
senza commettere una falsificazione oggettiva del dato.
$
\newline
$

$
In altre parole, il risultato di una misura è un numero privo di significato se
$
non accompagnato da una stima dell'{\bf errore}, perché una misura
$
identifica {\slshape sempre} un {\bf intervallo di valori} (e non un singolo valore numerico).
$
Per indicare correttamente questa proprietà, il risultato della misura dovrebbe essere
$
rappresentato da una forchetta, in uno di questi modi:
$

$
\begin{center}
$
\begin{math}
$
47,5 \frac{km}{h} < v < 52,5 \frac{km}{h}
$
\end{math}
$
oppure
$
\begin{math}
$
v = 50 \frac{km}{h} \pm 2,5 \frac{km}{h}
$
\end{math}
$
\end{center}
$