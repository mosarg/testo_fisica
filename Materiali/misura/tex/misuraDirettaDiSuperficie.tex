\subsubsection*{Una misura diretta di superficie}
$

$
L'immagine nel riquadro rappresenta il profilo di una superficie piana di forma irregolare.
$
Utilizzando un foglio di carta semitrasparente, acquisite un calco fedele del profilo ed appoggiatelo su un foglio
$
quadrettato in modo ragionevolmente sottile (per esempio, con un passo di 4 o 5 millimetri).
$
Ciascuno studente, conteggi diligentemente il numero dei quadretti completamente interni al proprio calco, scartando
$
quelli esterni e quelli che toccano anche in un solo punto il bordo.
$
\newline
$

$
Se avete operato in modo sufficiente attento e sufficientemente {\bfseries {\slshape indipendente}}, il numero dei
$
quadretti conteggiati sarà diverso per ciascun osservatore.
$
\newline
$

$
Discutete, ragionando tra di voi e con il vostro insegnante, sulle cause di questo fenomeno.
$
Contemporaneamente, usate questo esempio per riflettere sui problemi e le caratteristiche di un processo di misura.
$
\newline
$

$
Essenzialmente, la misura è una delle forme con le quali l'uomo, in fisica, osserva i fenomeni naturali. In modo
$
generico, si potrebbe sostenere che ogni affermazione fisica è il risultato di una misura nascosta. Per esempio, la
$
frase
$
{\slshape Tutti gli oggetti pesanti cadono verso il suolo}, nasconde la misura della direzione del moto di caduta e una
$
stima del peso necessario ai corpi per sottostare al comportamento descritto, distinguendoli
$
{\slshape quantitativamente} dai corpi leggeri. Tecnicamente, invece, la parola misura si riferisce all'operazione
$
con la quale viene associato un numero ad un determinato aspetto
$
misurabile di un fenomeno fisico (la {\slshape grandezza}). Questa operazione, tuttavia, non è sempre
$
immediata, come saremmo portati a pensare,  ma può riservare problemi e sorprese, come nell'esempio della nostra
$
superficie.
$
\newline
$

$
Se vogliamo cercare una definizione sintetica, potremmo dire che:
$

$
\begin{quote}
$
\bfseries{
$
una misura è il processo sperimentale con il quale un fisico determina alcune proprietà del sistema osservato, esprimendole in
$
termini matematici.
$
}
$
\end{quote}
$

$

$
