\subsection*{Definizioni}
$
\begin{itemize}
$
\item {\bfseries Misura}
$
Processo sperimentale che permette di descrivere alcuni aspetti caratteristici di un fenomeno, utilizzando strumenti
$
matematici.
$
\item {\bfseries Grandezza misurabile}:
$
Aspetto di determinato oggetto o fenomeno che può essere sottosposto a misura.
$
\item{\bfseries Grandezze omogenee}:
$
Si dicono omogenee due o più grandezze che possono essere sommate e confrontate tra loro.
$
\item{\bfseries Misura diretta}:
$
Processo sperimentale che permette di confrontare una grandezza misura con un campione della stessa specie
$
\item{\bfseries Misura indiretta}:
$
Processo sperimentale che permette di determinare una grandezza fisica attraverso un insieme di grandezze non omogenee.
$
\item{\bfseries Misura ripetibile}:
$
Processo di misura che produce risultati diversi ad ogni ripetizione. Ripetendo più volte una misura ripetibile ed
$
elaborando opportunamente i dati, è possibile migliorare la qualità della misura stessa.
$
\item{\bfseries Misura non ripetibile}:
$
Processo di misura strutturato in modo da ricavare lo stesso risultato ad ogni ripetizione. Il risultato di una misura
$
non ripetibile non è soggetto a miglioramenti.
$
\item{\bfseries Analisi dei dati, o statistica}:
$
Processo di carattere matematico per ricavare informazioni da un determinato campione relativo all'osservazione di una
$
data grandezza.
$
\item{\bfseries Campione statistico}:
$
Insieme dei risultati di una misura ripetibile.
$
\item{\bfseries Dispersione di una misura}:
$
Intervallo sul quale sono distribuiti i valori del campione. Spesso si determina sperimentalmente,
$
valutando il valore massimo e il valore minimo di un insieme di osservazioni.
$
\item{\bfseries Dispersione relativa}:
$
Rapporto tra la dispersione assoluta e il valore medio di un campione statistico.
$
\item{\bfseries Errore assoluto}:
$
Misura dell'intervallo con cui si ritiene di poter determinare il valore di una misura. Normalmente, l'errore assoluto
$
dovebbe essere più piccolo del corrispondente intervallo di disperiosne.
$
\item{\bfseries Valore atteso di un misura}:
$
Risultato di un misura, che corrisponde al valore ritenuto mediamente più probabli e per una misura.
$
\newline
$
Normalmente, nei problemi scolastici, corrisponde al valore medio.
$
\item{\bfseries Errore relativo}: Rapporto tra l'errore assoluto e il valore atteso di una misura.
$
\item{\bfseries Curva di frequenza}:
$
Istogramma che rappresenta il numero di eventi osservati per ogni sottoinsieme dell'intervallo di dispersione.
$
\item {\bfseries Valore quadratico medio}: 
$
\end{itemize}
$

$
