
$
\subsubsection*{Attività}
$

$
Immaginiamo di dover {\bfseries misurare un banco} .
$

$
Come possiamo procedere?
$

$
Ciascuno studente, in modo {\bfseries{\slshape indipendente}}, cioè senza consultare i compagni, annoti sul quaderno i
$
seguenti elementi:
$
\begin{itemize}
$
\item Gli strumenti necessari per misurare il banco;
$
\item Le operazioni principali del processo di misura;
$
\item Una stima del risultato atteso.
$
\end{itemize}
$
Al termine, confrontare le proposte con il contributo dell'insegnante.
$

$
\subsubsection*{Le grandezze fisiche}
$
Ogni fenomeno fisico coinvolge un insieme numeroso di proprietà distinte.
$
\newline
$

$
Un cavallo al galoppo occupa in ogni istante un punto diverso della pista. Per descriverne il moto è necessario
$
effettuare delle misure di {\bfseries tempo} e di {\bfseries spazio}.
$

$
Lungo la corsa, il cavallo solleva schizzi di fango dalle pozzanghere sul terreno. Da essi possiamo ricavare
$
informazioni sugli scambi di {\bfseries quantità di moto}, e distinguere, tra più animali, il {\bfseries più pesante} o
$
il {\bfseries più veloce}.
$

$
Ancora, il cavallo suda. Perciò aumenta la propria {\bfseries temperatura} corporea e scambia {\bfseries calore} con
$
l'ambiente.
$
\newline
$

$
Un tempo si può misurare in secondi, mesi o in anni. Una lunghezza può essere misurata in metri, chilometri, miglia o
$
in anni luce. Ogni misura descrive una grandezza riferendosi a un ben preciso elemento di paragone, chiamato unità di misura.
$
L'indicazione dell'unità di misura è un elemento indispensabile per dare senso compiuto a qualunque misura fisica.
$

$
Continuate la descrizione individuando nuove proprietà misurabili e ripetete il gioco per nuovi fenomeni indipendenti,
$
avendo cura di identificare sistematicamente la {\slshape proprietà misurabile} e il nome della {\slshape grandezza
$
fisica} corrispondente.
$

$
Al termine, confrontate le proposte dei singoli, per discuterle con l'insegnante.
$
