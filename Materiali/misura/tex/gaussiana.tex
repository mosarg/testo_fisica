\subsubsection*{La distribuzione gaussiana}
$
In questa esperienza, vogliamo osservare un fenomeno caratterizzato da errori distribuiti in modo {\slshape
$
normale}, secondo un meccanismo molto comune, che è stato studiato all'inizio dell'800 da Carl Friederich
$
Gauss\footnote{\url{http://it.wikipedia.org/wiki/Carl_Friederich_Gauss}}.
$
\newline
$
Un'attività che si presta bene a questo scopo può essere la misura del tempo di svuotamento di una bottiglia.
$

$
Procuratevi una bottiglia di plastica, producete un foro sul fondo e fissatela a testa in giù ad un rubinetto, senza
$
spostarla. Praticate anche un foro di piccole dimensioni sul tappo. Usando un pennarello indelebile, contrassegnate due
$
livelli di riferimento a dieci o quindici centimetri di distanza tra loro.
$

$
Lo scopo della misura sarà determinare il tempo necessario all'acqua per scendere dal livello superiore a quello
$
inferiore, durante lo svuotamento. Organizzatevi per acquisire il maggior numero possibile di misure manuali
$
indipendenti (diciamo parecchie {\itshape centinaia}) in un tempo accettabile. Per riuscirci sarà necessario
$
raccogliere più misure simultanee per ogni singolo svuotamento. Perciò, utilizzate, oltre al cronometro in dotazione
$
del vostro laboratorio, anche i vostri cronometri da polso o i palmari personali.
$

$
Prima di cominciare la presa dati, verificate le condizioni necessarie per effettuare un'acquisizione coerente
$
ed omogenea:
$
\begin{itemize}
$
\item Discutendo tra voi e con l'insegnate, definite bene le operazioni meccaniche di riempimento, avvio e stop della
$
misura, in modo da sentirvi sicuri di poterle ripetere sempre alo stesso modo con tutta la diligenza possibile;
$
\item Fate una stima grossolana dell'ordine di grandezza del tempo da misurare;
$
\item fate una stima grossolana della variabilità del campione (cioè della differenza che è possibile attendere tra due
$
misure indipendenti);
$
\item Calcolate il rapporto tra la stima di variabilità e la stima del tempo di svuotamento. Per ottenere una curva
$
gaussiana accettabile è importantissimo che questo valore sia sufficientemente piccolo. se non siete soddisfatti,
$
riprogettate il vostro metodo di lavoro studiando degli accorgimenti efficaci per aumentare il tempo di svuotamento o
$
per ridurre la variabilità di acquisizione;
$
\item per semplicità nella trattazione matematica, lavorate con strumenti che abbiano la stessa sensibilità assoluta
$
(sarebbe opportuno il centesimo di secondo), altrimenti, approssimate i valori degli strumenti troppo sensibili.
$
\end{itemize}
$

$
Raccogliete i dati in sequenza sul quaderno e, contemporaneamente, preparate un foglio di carta millimetrata per
$
istogrammarli immediatamente, durante l'acquisizione. Eventualmente, fate uso di strumenti elettronici.
$

$
Un istogramma di frequenza riporta in ascissa i valori misurati e in ordinata le frequenze per ogni singolo intervallo
$
di campionamento. Dall'osservazione dell'istogramma emerge il risultato della misura. Sebbene tutte le vostre
$
misure fossero indipendenti tra loro e affette da errori casuali, sui quali non potevate esercitare alcun controllo, vi
$
accorgerete che la distribuzione dei dati non risulta ugualmente causale. Infatti, vi sembrerà di osservare una curva
$
caratterizzata da:
$
\begin{itemize}
$
\item una particolare forma a campana, molto riconoscibile;
$
\item un picco che corrisponde abbastanza bene alla vostra stima iniziale;
$
\item due code simmetriche e molto basse, che si prolungano (idealmente) verso l'infinito.
$
\end{itemize}
$
Misurate la larghezza della campana a metà altezza. All'interno, dovreste raccogliere la maggior parte dei vostri dati.
$

$
In una curva gaussian\footnote{\url{http://it.wikipedia.org/wiki/Gaussiana}} l'ascissa del picco viene assunta come
$
valore centrale della misura (indichiamolo con \begin{math}\mu\end{math}), e la larghezza a metà altezza
$
(\begin{math}\sigma\end{math}) come stima dell'errore. Il vostro profilo sperimentale, inoltre, dovrebbe assomigliare
$
molto a questa funzione matematica, che descrive la probabilità di ottenere un valore \begin{math}x\end{math} diverso
$
dal valor medio:
$
\begin{equation}
$
P(x)~=~\frac{1}{\sqrt{2\pi\sigma^2}}e^{-\frac{{x-\mu}^2}{2\sigma^2}}
$
\end{equation}
$

$
Se ciò non fosse, dovreste ripensare un po' meglio ai vostri dati. Si possono presumere due cose situazioni diversi:
$
\begin{itemize}
$
\item la vostra distribuzione non assomiglia affatto a  quella descritta;
$
\item la vostra distribuzione assomiglia abbastanza a quella descritta, ma avete l'impressione che qualche singola
$
misura si discosti un po' troppo.
$
\end{itemize}
$
Nel primo caso significa che il vostro modo di lavorare (o l'esperimento in sé) nono sono descritti bene dalla
$
matematica proposta in questa attività. Nel secondo, date un occhio ai dati che si collocano nelle code della
$
gaussiana, molto lontano dal valore centrale.
$
La probabilità che una misura sperimentale si discosti dal valore centrale più di \begin{math} 3 \sigma\end{math} è
$
circa di 3 casi su
$
mille\footnote{\url{http://it.wikipedia.org/wiki/Funzione_di_ripartizione_della_variabile_casuale_normale}
$
}
$
. Se vi risulta di avere ottenuto un numero troppo elevato di eventi di coda,
$
rispetto al numero
$
dei dati raccolti, potete scartare i dati troppo lontani, e ricalcolare il valor medio e l'errore associato.
$