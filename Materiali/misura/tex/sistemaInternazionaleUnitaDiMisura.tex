\subsubsection*{Il sistema Internazionale delle Unità di Misura}
$
Lo spettro di fenomeni a cui la fisica si interessa è vastissimo, ed è altrettanto vasto
$
l'insieme delle grandezze osservabili necessarie per rappresentarlo.
$

$
La Comunità Scientifica ha ritenuto necessario stabilire un po' d'ordine sulla questione, incaricando
$
un organismo riconosciuto a livello internazionale di definire i nomi convenzionali delle grandezze,
$
i simboli relativi e le unità di grandezza standard a cui fare riferimento.
$
\newline
$

$
Questo orgnaismo si chiama Ufficio Internazionale dei Pesi e delle Misure (acronimo: {\bf BIPM}, è un'organizzazione senza
$
scopo di lucro, con sede in Francia, che ha questo sito di riferimento:
$
\newline
$

$
\url{http://www.bipm.org}
$
\newline
$

$
Come puoi verificare anche usando wikipedia\footnote{
$
\url{http://it.wikipedia.org/wiki/Sistema_internazionale_di_unit\%C3\%A0_di_misura}
$
\newline
$

$
}
$
il sistema internazionale (acronimo: {\bf SI}) individua un insieme limitato di unità di misura
$
fondamentali, che sono la {\bf lunghezza}, il {\bf tempo}, la {\bf massa} e l'{\bf intensità di corrente}.
$
\newline
$
Queste unità sono, rispettivamente il {\bf metro}, il {\bf secondo}, il {\bf chilogrammo massa} e
$
l'{\bf Ampere}.
$
\newline
$
\newpage
$
Per ciascuna di esse il {\bf BIPM} definisce un'unità di misura fondamentale, di cui stabilisce la definizione rigorosa.
$
\newline
$

$
Si tratta di una definizione soggetta a modificazioni nel tempo, inseguendo l'evoluzione del progresso tecnologico.
$
\newline
$
Attualmente\footnote{
$
\url{http://www.bipm.org/en/si/base_units/metre.html}
$
}:
$
\begin{center} \bf The metre is the length of the path travelled by light in vacuum during a time interval of 1/299
$
792 458 of a second.
$
\end{center}
$

$
La definizione originale dell'unità di misura del metro lineare, invece, possiede una storia affascinante\footnote{
$
\url{http://www.bipm.org/en/si/history-si/evolution_metre.html}
$
}.
$
\newline
$
Fu determinata, infatti, nel 1799, in piena rivoluzione francese, in funzione delle dimensioni della circonferenza
$
terrestre. Si stabilì, infatti, che un quarto dell'asse terrestre dovesse corrispondere a $10^7 m$ (dieci milioni di
$
metri). Siccome, in radianti, un quarto di circonferenza corrisponde a {\boldmath $\frac{\pi}{4}$}, rimane definita
$
automaticamente la misura del raggio terrestre, secondo la formula $\frac{C}{4}~=~\frac{\pi}{4}r~=6.3~10^3 km$.
$

$
L'iniziativa transalpina nella definizione delle unità di misura fondamentali fu oggetto di contestazioni per eccesso
$
di protagonismo da parte della comunità scientifica di area anglosassone, e non fu accettata completamente. Ancora
$
oggi, sebbene le convenzioni internazionali siano solidamente riconosciute ovunque, molte unità di misura del sistema
$
SI sono sostituite di fatto, nell'uso comune di molti stati nazionali che gravitano l'area del
$
Commonwealth Britannico, inclusi gli Stati Uniti, da unità di misura alternative\footnote{ecco uno dei tanti siti che
$
riportano una tabella di conversione tra unità standard e unità convenzionali:\newline
$
\url{http://www.oppo.it/tabelle/unita_misura_conversioni.htm}}
$
, come il pollice, il piede, la
$
iarda, il miglio (marino e terrestre), il gallone, la libbra, l'oncia ...
$
\newline
$

$
In fisica, l'uso delle unità di misura non standard non dovrebbe essere assolutamente {\itshape mai} usato, se non per
$
qualche salutare esercizio di conversione tra unità di misura.
$