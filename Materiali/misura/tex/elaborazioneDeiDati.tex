\subsection*{L'elaborazione dei dati}
$
Un fruttivendolo appoggia le mele sul piatto della bilancia e legge sul display peso e costo della merce.
$
\newline
$

$
La fisica non può accontentarsi di una rilevazione sommaria ed affrettata delle misure, ma le valuta criticamente in un
$
modo più attento. Nell'esempio della superficie irregolare, mostra una situazione
$
concreta nella quale un processo di misura produce, per un singolo oggetto, più risultati differenti. È compito
$
del fisico, in queste
$
situazioni, dare un senso all'informazione raccolta, superando l'apparente confusione iniziale.
$
Il processo che porta a questa
$
sintesi si chiama {\bfseries analisi dei dati}.
$
\newline
$

$
A volte, una buona analisi dei dati richiede procedure molto complesse. Nel nostro esempio, invece, possiamo individuare
$
un insieme di operazioni semplici e sensate che ci permettono di procedere. 
$
\begin{enumerate}
$
\item Ordinate tutti i risultati raccolti su una colonna, dal maggiore al minore, contrassegnando le eventuali misure
$
ripetute.
$
\item Calcolate la differenza tra il valore massimo e il valore minimo della misura. Il risultato può essere chiamato
$
{\bfseries dispersione (assoluta) della misura}.
$
\item Valutate se, a vostro giudizio, questo valore può essere ritenuto grande o piccolo. Un modo comune di fare questo
$
è dividere la dispersione assoluta della misura per uno dei risultati della misura stessa. In questo caso si parla di
$
{\bfseries dispersione relativa}.
$
\item Rappresentate i dati in una {\bfseries curva di frequenza}, realizzata come segue:.
$

$
Rappresentate in ascissa l'intervallo di dispersione dal valore minimo misurato a quello massimo e suddividetelo in un
$
piccolo numero di parti. Se avete venti misure, ad esempio, può bastare una suddivisione in quattro o cinque settori.
$

$
Per ogni settore riportate, in ordinata, il numero di ricorrenze della misura nell'intervallo corrispondente.
$

$
Discutete il grafico con l'insegnante, in particolare cercate di valutare se è possibile riconoscere qualche tipo di regolarità.
$
\item Riflettete sul grado di omogeneità dei vostri dati. Sicuramente, li avrete acquisiti usando sempre la stessa diligenza e la stessa
$
attenzione, tuttavia, in questo caso, non è possibile attribuire a tutti i dati lo stato grado di affidabilità. Scegliete ad esempio
$
due misure per difetto. Trattandosi necessariamente di valori più piccoli del valore vero della vostra superficie, potete riconoscere
$
con certezza quale delle due è più vicina alla misura cercata. Questo non significa necesariamente che l'operatore che ha ricavato il
$
dato sia stato più bravo, ma semplicemente più fortunato.
$

$
Da ciò, tuttavia, discende una conseguenza importante. Se i dati disponibili non hanno lo stesso grado di omogeneità, non sarà oportuno operare
$
su di essi una media matematica. Anzi, eseguendo la media dei dati, si otterrebbero valori peggiori di quelli già disponibili.
$
\item Se non è possibile effettuare una media matematica, significa che è stato inutile ripetere più volte la misura, o ritenete che la ripetizione abbia
$
comunque consentito di migliorare la misura finale? Esprimete la vostra opinione.
$
\item Discutendo tra voi e con l'insegnante, definite il modo più opportuno per eprimere il risultato finale del vostro lavoro.
$

$
\end{enumerate}
$
