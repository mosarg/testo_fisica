\subsubsection*{Le Misure indirette}
$
Preparate un recipiente di forma regolare (per esempio un parallelepipedo a base rettangolare) e riempitelo d'acqua
$
fino a circa metà. Prendete un sasso irregolare di dimensioni opportune e immergetelo nell'acqua. Osservando
$
l'innalzamento del livello, cercheremo di ricavare il volume del sasso e la precisione della misura.
$
\newline
$

$
In questo caso la situazione non si presta, come in precedenza, a una misura diretta, perché non possediamo nessun
$
campione di volume da utilizzare come unità di misura. Però, siccome la forma del recipiente è regolare, possiamo
$
ricavare il volume da un insieme di misure di lunghezza. Quando una grandezza viene ricavata da misure di grandezze
$
diverse, non omogenee, attraverso una serie di calcoli matematici, si dice di aver eseguito una misura indiretta.
$

$
Prima di tutto, ricaviamo la superficie del rettangolo d'acqua dalla misura dei suoi lati, diciamo ${\mathbf
$
{\overline{l_1}}}$ e ${\mathbf {\overline{l_2}}}$. Accontentiamoci di eseguire una sola misura non ripetibile, ma
$
teniamo conto con attenzione del corrispondente errore di misura, scrivendo quindi:
$
\begin{center}
$
\begin{eqnarray}
$
l_1~=~\overline{l_1}~\pm~\Delta l_1 \\
$
l_2~=~\overline{l_2}~\pm~\Delta l_2 \nonumber
$
\end{eqnarray}
$
\end{center}
$

$
Prima di determinare il valore di $\overline S$, osserviamo la seguente figura:
$

$
  \begin{figure}[H]
$
  \centering
$
   \includegraphics[width=\textwidth]{../immagini/propagazioneDegliErrori.png}
$
   \label{figure:propagazioneDegliErrori}
$
   \caption{La propagazione delgli errori}
$
  \end{figure}
$

$
Nella figura, la superficie misurata è quella tratteggiata. gli errori di misura possono spostare il vertice in alto a
$
sinistra in un punto qualunque del rettangolino verde. Quindi, il più piccolo valore possibile per la misura è:
$
\begin{center}
$
\begin{math}
$
S_{min}~=~(\overline{l_1}~-~\Delta l_1)(\overline{l_2}~-~\Delta l_2)
$
\end{math}
$
\end{center}
$
e il maggiore:
$
\begin{center}
$
\begin{math}
$
S_{max}~=~(\overline{l_1}~+~\Delta l_1)(\overline{l_2}~+~\Delta l_2)
$
\end{math}
$
\end{center}
$

$
Il valore centrale, invece, è semplicemente:
$
\begin{center}
$
\begin{math}
$
\overline{S}~=~\frac{S_{min}+S_{max}}{2}~=~\overline{l_1}~\overline{l_2}
$
\end{math}
$
\end{center}
$

$
Prima di definire l'errore, osserviamo meglio la figura. A rigore, l'errore corrisponde alla fascia esterna, composta
$
da tre rettangoli. Quello centrale, però è molto più piccolo degli altri. Infatti si ottiene moltiplicando tra loro i
$
due errori assoluti. In matematica, la moltiplicazione di due quantità piccole è una quantità estremamente piccoli.
$
Spesso, il rettangolo rosso è così piccolo che è possibile trascurabile del tutto, nella stima dell'errore su $S$.
$
Perciò scriveremo:
$
\begin{center}
$
\begin{math}
$
\Delta S~=~\overline{l_1}~\Delta l_2~+\overline{l_2}~\Delta l_1
$
\end{math}
$
\begin{equation}
$
S~=~\overline{S}~\pm~\Delta S~=~\overline{l_1}~\overline{l_2}~\pm~(\overline{l_1}~\Delta l_2~+~\overline{l_2}~\Delta l_1)
$
\end{equation}
$
\end{center}
$
A questo punto, possiamo passare allo studio del volume. Anche in questo caso, ci troviamo davanti ad una
$
moltiplicazione.
$
\begin{center}
$
\begin{math}
$
V~=S~h
$
\end{math}
$
\end{center}
$
Se la struttura matematica è la stessa, procederemo allo stesso modo di prima:
$
\begin{center}
$
\begin{eqnarray}
$
\overline{V}~=~\overline{h}~\overline{S} \nonumber \\
$
\Delta V~=~\overline{h}~\Delta S~+~\overline{S}~\Delta h
$
\end{eqnarray}
$
\end{center}
$
Prima di concludere, effettuamo il calcolo degli errori relativi.
$
\newline
$
Per la superficie:
$
\begin{center}
$
\begin{equation}
$
\epsilon_S~=~\frac{S}{\overline{S}}
$
=\frac{\overline{l_1}~\Delta l_2~+~\overline{l_2}~\Delta l_1}{\overline{l_1}*~\overline{l_2}}
$
~=~\frac{\Delta l_2}{\overline {l_2}}~+~\frac{\Delta l_1}{\overline{l_1}}
$
\end{equation}
$
\end{center}
$
Per il volume, invece:
$
\begin{center}
$
\begin{equation}
$
\epsilon_V~=~\frac{V}{\overline{V}}~=~
$
\frac{\overline{S}~\Delta h~+~h~\Delta S}{S~h}
$
~=~\frac{\Delta h}{\overline{h}}~+~\frac{\Delta l_2}{\overline{l_2}}~+~\frac{\Delta l_1}{\overline{l_1}}
$
\end{equation}
$
\end{center}
$
Vi sarete accorti che le formule hanno un aspetto molto simile.
$
\newline
$
L'errore relativo sulle superfici si ottiene sommando gli errori relativi delle misure di lunghezza per i lati della
$
superficie. L'errore relativo nella misura di volume, si ottiene aggiungendo ancora l'errore relativo per la dimensione
$
verticale.
$
\newline
$

$
In conclusione, se una grandezza derivata si calcola per mezzo di una moltiplicazione di fattori, l'errore relativo
$
per la grandezza derivata risulta uguale alla somma degli errori relativi di ciascun fattore, presi singolarmente.
$
Questo assunto è un caso particolare di una teoria più ampia, chiamata "Legge di propagazione degli errori".
$
