\documentclass[a4paper,10pt,oneside]{article}
%\newcounter{exercise}
%\newenvironment{exercise}%
%{\refstepcounter{exercise}\vspace{10pt}\par\noindent
%\textbf{Esercizio \theexercise\ }
%\begin{itshape}\par\noindent}
%{\end{itshape}} 
\newtheorem{es}{Esercizio}
\usepackage[margin=3cm,noheadfoot]{geometry}
\renewcommand{\top}{0cm}
\renewcommand{\floatpagefraction}{0.98}
\renewcommand{\topfraction}{1}
\usepackage{amsfonts}
\usepackage{amsmath}
\usepackage{amscd}
\usepackage{amssymb}
\usepackage[italian]{babel}
\usepackage[utf8x]{inputenc}
\usepackage{mathrsfs}
\thispagestyle{empty}
\begin{document}

{\Large La legge di Biot e Savart}\\


\vspace{3cm}
Come possiamo calcolare l'intensità del campo magnetico prodotto da un elemento infinitesimo $Id\mathbf{l}$ di corrente in un punto $\mathbf{r}$?
Dallo studio della forza di Lorentz siamo riusciti a dedurre che la forza infinitesima esercitata da un campo magnetico \footnote{generato da un magnete permanente o da una distribuzione di corrente}$\mathbf{B'}$ su un elemento di corrente $d\mathbf{l}$ è:
\begin{equation}
 d\mathbf{F'}=Id\mathbf{l}\times \mathbf{B'}
\end{equation}
in virtù della terza legge di Newton tale forza deve essere uguale e contraria a quella esercitata dall'elemento di corrente sulla sorgente del campo $\mathbf{B'}$. Immaginiamo che ad una distanza $\mathbf{r}$ dall'elemento di corrente $Id\mathbf{l}$ sia posizionato un polo magnetico positivo unitario (possiamo immaginare che tale polo appartenga ad una barra magnetica molto lunga).
Per la terza legge della dinamica la forza di reazione dovrà essere allora:
\begin{equation}\label{reazione}
 d\mathbf{F}=-d\mathbf{F'}=-Id\mathbf{l}\times \mathbf{B}'
\end{equation}
Dove $\mathbf{B}'$ è il campo magnetico creato dal polo magnetico in $\mathbf{r}$ e $d\mathbf{F}$ è la forza esercitata dall'elemento di corrente sul magnete. Dalla definizione dell'intensità magnetica $\mathbf{H}$ possiamo scrivere:
\begin{equation}\label{defi}
 \mu_0\mathbf{H}'=\mathbf{B}'=\frac{1}{4\pi}\left(-\frac{\mathbf{r}}{r^3}\right)
\end{equation}
La forza $d\mathbf{F}$ esercitata dall'elemento di corrente sul polo unitario in $\mathbf{r}$ è per definizione l'intensità magnetica $d\mathbf{H}$ generata dalla corrente in tale punto possiamo quindi scrivere:
\begin{equation}
 d\mathbf{H}=d\mathbf{F}
\end{equation}
manipolando leggermente la [\ref{reazione}] otteniamo (ricordiamo che $\mathbf{B}=\mu_0\mathbf{H}$):
\begin{equation}\label{fine1}
 d\mathbf{B}=\mu_0d\mathbf{H}=-\mu_0Id\mathbf{l}\times \mathbf{B}'
\end{equation}
sostituendo ora la [\ref{defi}] nella [\ref{fine1}] otteniamo:
\begin{equation}
 d\mathbf{B}=\frac{\mu_0I}{4\pi}\left(\frac{d\mathbf{l}\times \mathbf{r}}{r^3}\right)
\end{equation}



\end{document}
