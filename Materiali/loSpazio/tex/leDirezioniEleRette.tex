\subsubsection*{Le direzioni e le rette}

Un'operazione molto importante, per muoversi nello spazio, è saper controllare la direzioni.\newline
Come esercizio introduttivo, cominciamo a rappresentare sul piano cartesiano una successione di punti con la procedura sotto descritta:
\begin{enumerate}
\item {scegliamo un punto iniziale a piacere. Per esempio il punto $\mathbf {(-2;-5)}$;}
\item {partendo dal punto precedente, spostiamoci di due unità in orizzontale, senza segnare alcun punto;}
\item {di seguito, spostiamoci di tre unità orizzontale e segniamo un secondo punto;}
\item {ripetiamo le operazioni precedenti dal passo 2.}
\end{enumerate}
Sarà facile verificare che, in questo modo, si ottiene una successione di punti allineati lungo una retta.\newline

Perché proprio una retta? In verità, non è stato per caso, ma ciò che è accaduto è un fenomeno ben riproducibile.\newline
Tracciate infatti una retta inclinata su un foglio bianco, ben squadrato.\newline
Contrassegnate, su questa retta, cinque o sei punti, non necessariamente equidistanti e nominateli indicizzandoli opportunamente \footnote{chiamateli ad esempio $P_1=(x_1;y_1),P_2=(x_2;y_1),...$ }.\newline
Scegliete tre o quattro coppie di questi punti, non necessariamente consecutivi.\newline
Per ogni coppia di punti, costruite dei triangoli rettangoli, con i cateti paralleli ai bordi del foglio e l'ipotenusa aderente alla retta data.

Vi accorgerete facilmente di aver rappresentato una famiglia di triangoli simili.\newline
Chiamate $\Delta x_i$ i cateti orizzontali di ciascun triangolo e $\Delta y_i$ i corrispondenti cateti verticali.
Se conoscete le proprietà dei triangoli simili, potrete dedurre facilmente che risulta:
\begin{center}
\begin{math}
\frac{\Delta y_1}{\Delta x_1} = \frac{\Delta y_2}{\Delta x_2} = ...
\end{math}
\end{center}

Questo significa che il rapporto $\frac{\Delta y}{\Delta x}$ è una {\bfseries \slshape {propietà della retta}}.
Per questa ragione, viene chiamato {\bfseries coefficiente angolare} e frequentemente indicato con la lettera {\slshape m}:
\begin{center}
\begin{equation}
m = \frac{\Delta y}{\Delta x}
\end{equation}
\end{center}

Come esercizio, provate a generare una successione di rette con la tecnica indicata all'inizio di questo paragrafo e a valutare il corrispondente coefficiente angolare.\newline
Provate anche a tracciare alcune rette con valori predeterminati di {\slshape m}. Per esempio: {\slshape m} $= 1$ {\slshape m}$ = 2$, {\slshape m} $= 3$, {\slshape m}$ = -2$.\newline
Cosa potete osservare?
\newline

Cerchiamo ora di raccogliere in un modo sistematico le conoscenze fin qui acquisite.
\newline
Osserviamo, innanzitutto che, per determinare una retta, è sufficiente conoscere un suo punto, che possiamo chiamare $\mathbf {(x_0;y_0)}$ e il coefficiente angolare $\mathbf m$.
\newline
Desideriamo trovare un metodo automatico per individuare ogni altro punto della retta.
\newline

Riscriviamo in questo modo il concetto di coefficiente angolare:
\newline
\begin{center}
\begin{math}
\Delta_y = m \Delta_x
\end{math}
\end{center}
Ora, per ogni punto generico $P = (x;y)$ della retta data, risulta:
\begin{center}
\begin{math}
\Delta_y = y - y_0
\end{math}
\end{center}
\begin{center}
\begin{math}
\Delta_x = x - x_0
\end{math}
\end{center}
Quindi:
\begin{center}
\begin{equation}\label{eq:equazioneDellaRetta}
y = y_0 + m (x -x_0)
\end{equation}
\end{center}
Quest'ultima espressione è detta equazione della retta e permette di ricavare l'ordinata di ciascun punto in funzione della propria ascissa.
