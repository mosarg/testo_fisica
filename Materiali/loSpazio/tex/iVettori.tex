\subsection*{I vettori e le dimensioni dello spazio}
La geometria euclidea è una teoria affascinante, perché descrive lo spazio senza usare nessuna struttura di supporto.

Nelle applicazioni fisiche, però, può essere utile costruire un linguaggio più versatile, che si adatta più facilmente alle situazioni dinamiche dei fenomeni in evoluzione. Per questo motivo, molto spesso, si usano degli oggetti particolari, chiamati vettori.
\newline

Un vettore non è molto dissimile da un semplice punto dello spazio, ma aggiunge ad esso una struttura che aiuta a riflettere sulle relazioni tra quel punto e un altro punto particolare, che funge da riferimento. Per questa ragione, ogni vettore possiede un {\bfseries punto di applicazione} e un {\bfseries estremo}. Un vettore si disegna come una freccia che parte nel punto di applicazione e finisce nel punto di estremo.\newline
Nel linguaggio simbolico, il vettore è indicato da un nome soprassegnato da una freccia, in questo modo: $\vec v$\newline
Molti testi, tuttavia, usano indicare abitualmente i vettori con un semplice simbolo in grassetto. Così: $ \mathbf v$ .
\newline

Ogni vettore può essere determinato indicato per mezzo di tre caratteristiche specifiche. Il {\bfseries modulo}\footnote {o intensità} , la {\bfseries direzione} e il {\bfseries verso}.\newline
Il {\slshape modulo} di un vettore è la distanza tra il vertice e il punto di applicazione.\newline
La {\slshape direzione} è la retta passante per il vertice e la coda del vettore.\newline
Il {\slshape verso} è l'orientazione del vettore, ed è diretto dalla coda verso la punta.
\newline

Due vettori che possiedono la stessa direzione, lo stesso verso e la stessa intensità sono detti equivalenti. Per riconoscere due vettori equivalenti è sufficiente sovrapporli uno sull'altro, in modo che le rispettive punte e le rispettive code coincidano perfettamente, per mezzo di una traslazione rigida \footnote {Dunque una traslazione è rigida se conserva modulo, direzione verso del vettore}.
\newline

Come per ogni concetto della matematica, è fondamentale riconoscere le operazioni concrete che si possono definire sui vettori, per poter dire di aver capito in modo sufficiente l'argomento.\newline
Queste, per la verità, sono molto numerose, ma le principali, per i nostri scopi introduttivi, sono la {\bfseries somma}, la {\bfseries differenza} e la {\bfseries scomposizione}.\newline

Per sommare due vettori tra loro, è sufficiente trasportare, con una traslazione rigida, la coda del primo dei due sul vertice del secondo. Il vettore somma è uguale a quel vettore applicato nel punto di applicazione del secondo vettore e terminato nel punto di estremo del primo vettore \footnote {vedi disegno da integrare}. Come la geometria euclidea spiega bene, l'effetto di una somma di vettori rappresenta la diagonale del quadrilatero costruito sui i due vettori addendo.\newline
\newline

La {\slshape differenza} è l'operazione inversa della somma. In matematica, per affermare che 7 meno 3 fa 4, si deve prima verificare che 4 + 3 faccia 7. Un secondo modo molto efficente per calcolare una differenza è di applicare la tecnica del resto dal rigattiere:\newline
Se, dovendo acquistare un prodotto che vale, ad esempio, venti euro, un cliente porge una banconota da cinquanta, il negoziante calcola il resto contando a partire da venti, fino ad arrivare a cinquanta.
Allo stesso modo, per fare una differenza tra un vettore  dato (diciamo $\vec v_1$) e un secondo ($\vec v_2)$, è sufficiente tracciare un terzo vettore ($\vec {d}$) applicato sull'estremo di $\vec v_2$ e terminato su $\vec v_1$.\newline
Per verificare di avere ottenuto la differenza corretta, si può sommare $\vec v_2$ con $\vec {d}$ e constatare che si riottiene di nuovo il vettore iniziale $\vec v_1$.\newline
La differenza tra vettori è uno strumento molto efficace per rappresentare lo {\slshape spostamento} tra due punti.\newline
Osservando il parellogramma costruito su una coppia di vettori, si può osservare che la loro differenza corriponde alla sua diagonale trasversa, e biseca vicendevolmente il vettore somma.
\newline

La {\slshape scomposizione}, invece, è l'operazione che permette ai vettori di rappresentare le dimensioni dello spazio.\newline
Lo spazio della geometria, infatti, è un ente tridimensionale, che si estende in lunghezza, larghezza e profondità.\newline
Tuttavia, per gli scopi di questo manuale, tutte le applicazioni saranno ridotte alle due dimensioni dello spazio piano, salvo diversa indicazione.
\newline

Date due rette incidenti, che vengono dette anche {\bfseries \slshape {direttrici}} del sistema, e un vettore del piano, applicato nel punto di intersezione delle due, si dice scomposizione dei due vettori quella coppia di vettori appartenenti alla direttrice che, sommati tra loro, ricostituiscono il vettore iniziale. I vettori della coppia si dicono {\bfseries \slshape{componenti}} del vettore iniziale.\newline
La scomposizione di un vettore del piano è {\bfseries unica}. Infatti, dato un sistema di rette direttrici, esiste una e soltanto una coppia di componenti per ciascun vettore del piano. Se provassimo a scomporre un vettore del piano con l'uso di tre direttrici, anziché due, ci accorgeremmo che la scompozione non è unica. Per questa ragione, si dice che il piano è un ente a due dimensioni.
\newline

Con l'aiuto dell'insegnante, mostrate che la scomposizione di un vettore del piano su tre direttrici, anziché due sole, è un operazione che non produce un risultato unico, ma può essere realizzata con un numero infinito di terne differenti.
