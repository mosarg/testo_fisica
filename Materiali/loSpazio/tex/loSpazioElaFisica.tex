\subsubsection*{Lo Spazio, tra Fisica e Geometria}

Molti storici della Fisica sostengono che i primi studiosi ad occuparsi di Fisica in un modo paragonabile a quello dei nostri scienziati contemporanei siano stati i filosi ellenistici della Grecia antica.
Infatti, per loro, la cosidetta Geometria Euclide era, al tempo stesso, una {\bfseries teoria assiomatica} \footnote{elaborazione matematica fondata su un insieme di ipotesi astratte} e un'attività concreta di esplorazione delle proprietà dello spazio fisico. Per accettare la dimostrazione di un teorema i Greci {\slshape pretendevano} che il procedimento potesse essere visualizzato facendo uso di qualche strumento concreto, ed erano talmente fedeli a questa regola operativa che la loro è stata chiamata anche ``Geometria con riga e compasso''. 
\newline

Anche oggi, la Fisica mantiene questo atteggiamento, cercando di stabilire un confronto tra gli enti astratti della Geometria e le proprietà dello spazio reale. Per fare un esempio, se si domandasse ad un Fisico moderno di indicare un oggetto simile ad una retta, egli suggerirebbe quasi sicuramente di pensare ad un raggio di luce che viaggia nel vuoto.
\newline

Non sempre, però, il comportamento della luce si adegua alle regole della Geometria Euclidea. Esistono alcuni fenomeni nei quali la luce segue percorsi un po' sorprendenti. Consideriamo ad esempio il fenomeno dell'eclissi totale di Sole.
\newline
Durante un'eclissi totale, il disco solare non viene mai oscurato completamente, ma rimane visibile una corona luminosa molto spettacolare. Bisogna sapere, però, che il disco solare ha una dimensione più piccola di quello lunare. Ovvero, se facessimo una fotografia del sole, \footnote {usando schermi adeguati per proteggere la vista} e la confrontassimo con quella della Luna, la seconda risulterebbe più grande. Questo accade perchè i raggi solari, passando vicino alla luna, curvano assecondando la forza di gravità.
\newline

Riflettendo su questo fenomeno, Albert Einstein osservò che l'eclisse manifesta una {\bfseries {\slshape violazione}} della natura euclidea dello spazio. Infatti, un raggio di luce che, partendo dal centro del Sole fosse diretto verso il centro della Terra durante l'eclisse, potrebbe percorrere un numero infinito di strade diverse, contro il principio fondamentale per cui, {\slshape per due punti passa una e una sola retta}. 
\newline

Attualmente, la geometria euclidea è utilizzata ancora per descrivere un grande numero di fenomeni fisici e deve essere quindi ben conosciuta dagli studenti, ma è importante studiarla tenendo presente che tutte le costruzioni della matematica sono utili nel limite in cui aiutano gli uomini a descrivere in modo efficente i fenomeni osservati in concreto.
