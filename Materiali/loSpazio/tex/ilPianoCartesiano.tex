\subsection*{Il Piano cartesiano}

Sopra, abbiamo parlato soprattutto di punti e di rette.
\newline

Come potremmo, invece, riconoscere un piano?
\newline

Una definizione geometrica comune è la seguente:

"Date due rette incidenti, che possono essere chiamate rette generatrici, si dice piano l'insieme di tutte le rette che
possono essere costruite utilizzando le coppie di punti appartenenti alle generatrici."
\newline

Questa definizione è costruttiva. Infatti usa due concetti semplici (le rette generatrici) per costruire un concetto
nuovo. Usando questo modo di lavorare, si può arrivare alla definizione di uno strumento molto utile per rappresentare
lo spazio, che è chiamato piano cartesiano.
\newline

Un piano cartesiano fa uso di una coppia di rette ortogonali (assi coordinati), chiamate ascissa (asse orizzontale) e
ordinata (asse verticale). I punti di ciascun asse vengono marcati numericamente, associando ogni punto di ciascun asse
a un numero,
detto coordinata.
\newline

Il punto di intersezione delle generatrici viene chiamato origine. Normalmente, l'origine è associata alla
coordinata zero sia sull'asse delle ascisse che sull'asse delle ordinate.
\newline

Su ciascun asse viene definita una opportuna unità, marcando con il numero uno un punto diverso dall'origine.\newline
Successivamente, tutti gli altri punti dei due assi vengono marcati secondo un opportuno ordinamento compatto.
\newline

Finalmente, è possibile marcare tutti i punti del piano generato dagli assi ccoordinati, associando ciscun punto a una
coppia di numeri. Normalmente, il primo numero si ottiene proiettando il punto dato sull'asse delle ascisse e il
secondo proiettando lo stesso punto sull'asse delle ordinate, nel modo rappresentato in figura.
\newline
