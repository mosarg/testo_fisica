\subsubsection*{Il piano Cartesiano}

Come abbiamo visto in precedenza, la scomposizione di un vettore è un operazione applicabile a qualunque sistema di assi, ma di solito è comodo usare una coppia di {\bfseries rette ortogonali}. Se invece si lavora sull'insieme dei punti dello spazio, sarà necessario usare tre direttrici, che prendono il nome di {\bfseries terna cartesiana} \footnote{oppure terna ortogonale}.\newline
Un sistema di assi cartesiani conferisce allo spazio una struttura di riferimento, che è di grande utilità per eseguire operazioni sui vettori.
 
Ogni vettore $ \vec{v}$ del piano, infatti, può essere scomposto in un modo unico in due componenti, in modo che sia:
\begin{center}
\begin{math}
\vec {v} = \vec{v_x} + \vec{v_y}
\end{math}
\end{center}
Ciascuna componente di un dato vettore è, a sua volta un vettore, la cui intensità è determinata in funzione di una corrispondente unità di misura.\newline
Su ciascun asse cartesiano, pertanto, viene definito un vettore unitario, detto {\bfseries versore}.
Comunemente, i versori dello spazio tridimensionale sono indicati dalle lettere
$\vec i$,$\vec j$ e ${\vec k}$. Si può scrivere dunque:
\begin{center}
\begin{equation}\label{eq:scomposizioneDiUnVettore}
\vec v = v_x \vec i + v_y \vec j
\end{equation}
\end{center}

Usare il piano cartesiano rende più semplici le operazioni di somma e differenza. Infatti è possibile operare separatamente sulle componenti cartesiane indipendenti, come nel seguente esempio:
\begin {center}
\begin{math}
\vec v \pm \vec u = (v_x \vec i \pm v_y \vec y) + (u_x \vec i \pm u_y \vec j) = (v_x \pm u_x) \vec i + (v_y \pm u_y) \vec j
\end{math}
\end{center}

Apparentemente, le operazioni di somma e differenza sui vettori lavorano come le corrispondenti operazioni sui numeri reali.\newline
A volte, però, si possono avere delle piccole sorprese, come in questo esempio:
\begin {center}
\begin {math}
\vec v = 8 \vec i + 2 \vec j
\end{math}
\end{center}

\begin {center}
\begin {math}
\vec u = -8 \vec i + 2 \vec j
\end{math}
\end{center}
Determinate la somma e la differenza di questi due vettori. Cosa osservate?
\newline

In ultimo, osserviamo che, se sono note le coordinate cartesiane di un vettore, è sempre possibile ricavarne l'intensità, usando il teorema di Pitagora:
\begin{center}
\begin {equation}\label{eq:intensitaDiUnVettore}
v=\sqrt{{v_x}^2+{v_y}^2}
\end{equation}
\end{center}
